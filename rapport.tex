\documentclass[a4paper]{book}
\usepackage{fullpage}

\usepackage[utf8]{inputenc}
\usepackage[T1]{fontenc}
\usepackage[francais]{babel}

\usepackage{latexsym}
\usepackage{fancyhdr}
\usepackage{makeidx}
\usepackage{graphics}
\usepackage{graphicx}
\usepackage{longtable}
\usepackage{moreverb}
\usepackage{listings}

\newcommand{\altarica}{{\sc AltaRica}}

\begin{document}

\title{Master 1, Conceptions Formelles\\
Projet du module \altarica\\
Synthèse (assistée) d'un contrôleur du niveau d'une cuve}

\date{}

\author{CATALDI \and ERRARD}

\maketitle

\chapter{Le sujet}
\input{tank}

\chapter{Le rapport}
Le rapport est sur 20 points.

\section{Rôle du fichier {\tt GNUmakefile} (2 points)}
Les fichiers GNUmakefile permettent de générer différents contrôleurs (classique et optimisés, avec et sans vannes virtuelles).
Les différents contrôleurs sont donc générer en fonction d'un nombre maximum d'erreurs possible (allant de 0 à 3).
On optimise ensuite ses controleurs en itérant 5 fois sur le contrôleur précédent générer.
Il retourne également les résultats des spécification pour chacun des contrôleurs générer.

\section{Rôle de la constante {\tt nbFailures} et de l'assertion associée (1 point)}
La constante {\tt nbFailures} permet de représenter le nombre maximum d'erreur du système possible.
L'assertion associée permet donc de définir que le nombre d'erreurs du système ne peux pas dépasser nbFailures. 

\section{Résultats avec le contrôleur initial {\tt Ctrl}}
\subsection{Calcul d'un contrôleur}
\subsubsection{Avec 0 défaillance (0.5 point)}
\lstinputlisting{Res/System0FCtrl.res}
\lstinputlisting{Res/System0FCtrl0F1I.res}
\lstinputlisting{Res/System0FCtrl0F2I.res}
%\lstinputlisting{Res/System0FCtrl0F3I.res}
%\lstinputlisting{Res/System0FCtrl0F4I.res}
\paragraph{Interprétation des résultats}
\begin{flushleft}
    On constate que pour un système avec 0 erreur, on constate qu'a partir de la 1 ème itération on diminue de 153 le nombre d'état total
    et de 3225 le nombre de transition. On obtient également plus aucun état dans un niveau critique ou une situation redouté.
    De plus dans le  contrôleur de base, environ 33 \% des transitions sont des coups gagnants.
    A partir de la première itération 83\% des transition sont gagnant.
    
    On remarque également qu'il n'y a aucune différences entre le résultats à la 1ère et à la deuxième itération sont identique.
    On peut donc déterminé qu'on a générer le contrôleur optimal au bout de la première itération. 
        
\end{flushleft}

\subsubsection{Avec 1 défaillance (0.5 point)}
\lstinputlisting{Res/System1FCtrl.res}
\lstinputlisting{Res/System1FCtrl1F1I.res}
\lstinputlisting{Res/System1FCtrl1F2I.res}
\lstinputlisting{Res/System1FCtrl1F3I.res}
%\lstinputlisting{Res/System1FCtrl1F4I.res}
\paragraph{Interprétation des résultats}
\begin{flushleft}
    On constate que pour un système avec 1 erreur, on constate que entre le controleur de base et la dernière itération on diminue de 450 le nombre d'état total
    et de  14 379 le nombre de transition. On obtient également 260  état de moins dans un niveau critique et 233 état de moins dans une situation redouté.
    Mais le nombre de d'état puis est passé de 0 à 96. 
    De plus dans le  contrôleur de base, environ 25 \% des transitions sont des coups gagnants.
    A la dernière itération 56\% des transition sont gagnant.
    
    On remarque qu'on obtient le nombre optimal d'état dès la première itération. Et qu'au fil des itération on augmente le pourcentage de coup gagnant mais on augmente aussi le nombre de situation redouté.
    
\end{flushleft}

\subsubsection{Avec 2 défaillances (0.5 point)}
\lstinputlisting{Res/System2FCtrl.res}
\lstinputlisting{Res/System2FCtrl2F1I.res}
\lstinputlisting{Res/System2FCtrl2F2I.res}
\lstinputlisting{Res/System2FCtrl2F3I.res}
\lstinputlisting{Res/System2FCtrl2F4I.res}
\paragraph{Interprétation des résultats}
\begin{flushleft}
    On constate que pour un système avec 2 erreurs, on constate que entre le controleur de base et la dernière itération on diminue de 855 le nombre d'état total
    et de  38 074 le nombre de transition. On obtient également 444 état de moins dans un niveau critique et 281 état de moins dans une situation redouté.
    Mais le nombre de d'état puis est passé de 0 à 270. 
    De plus dans le  contrôleur de base, environ 16\% des transitions sont des coups gagnants.
    A la dernière itération 43\% des transition sont gagnant.
    
    On constate qu'on obtient le contrôleur le plus optimal à partir de la 3ème itération car aucune des valeur ne change.
    
\end{flushleft}

\subsubsection{Avec 3 défaillances (0.5 point)}
\lstinputlisting{Res/System3FCtrl.res}
\lstinputlisting{Res/System3FCtrl3F1I.res}
\lstinputlisting{Res/System3FCtrl3F2I.res}
\lstinputlisting{Res/System3FCtrl3F3I.res}
%\lstinputlisting{Res/System3FCtrl3F4I.res}
\paragraph{Interprétation des résultats}
\begin{flushleft}
    On constate que pour un système avec 3 erreurs, on constate que entre le controleur de base et la dernière itération on diminue de 855 le nombre d'état total
    et de  38 074 le nombre de transition. On obtient également 1770 état de moins dans un niveau critique et 281 état de moins dans une situation redouté.
    Mais le nombre de d'état puis est passé de 0 à 270. 
    De plus dans le  contrôleur de base, environ 16\% des transitions sont des coups gagnants.
    A la dernière itération 43\% des transition sont gagnant.
    
    On constate qu'on obtient le contrôleur le plus optimal à partir de la 3ème itération car aucune des valeur ne change.
    
\end{flushleft}

\subsection{Calcul des contrôleurs optimisés (2 points)}
\lstinputlisting{ControleursOpt/Optimisation.alt}
% A COMPLETER en expliquant en quoi consiste l'optimisation mise en place.

\paragraph{Avec 0 défaillance}
\lstinputlisting{Res/System0FCtrl0F2I_Opt.res}
% A COMPLETER en analysant les contrôleurs optimisés obtenus.

\section{Construction d'un contrôleur initial plus performant}
\subsection{Rôle du composant {\tt ValveVirtual}(2 points)}
\begin{flushleft}
    Le composant ValveVirtual est une alternative au composant Valve présent dans Ctrl.
    La ValveVirtual possède non seulement un rate mais aussi un rateReal, le premier étant modifier par la valve et le second par le contrôleur.
    Contrairement à la Valve classique qui utilise un booléen pour savoir si elle est coincé, celle-ci compare sa valeur rate et rateReal, la ValveVirtual utilise ces deux valeurs comme garde de ses transitions, si les deux valeurs sont les mêmes alors la ValveVirtual n'est pas coincé et peut donc effectuer des transitions, si les deux valeurs sont différentes alors la ValveVirtual ne fait rien.

\end{flushleft}

\subsection{Rôle du composant {\tt CtrlVV} (5 points)}
% A COMPLETER en expliquant les mécanismes mis en oeuvre, leurs rôles et les avantages de ce contrôleur par rapport au précédent CtrlVV.

\begin{flushleft}
    Le composant CtrlVV est une alternative au Ctrl qui au lieu d'utiliser des Valve, utilise des ValveVirtual.
    Ce composant vérifie par le biais d'assertion que les valeurs de ses rate et les rateReal de ses ValveVirtual soient les mêmes.
    Ces assertions, un fois combiner avec les gardes des ValveVirtual garantisse que si tout va bien alors les valeurs des rate des ValveVirtual et celle du CtrlVV soient les mêmes.
    En plus, à la différences du Ctrl, le CtrlVV fait aussi la synchronisation entre les différentes ValveVirtual lors des actions du contrôleur.

\end{flushleft}

\section{Résultats avec le contrôleur {\tt CtrlVV}}
\subsection{Calcul d'un contrôleur}
\subsubsection{Avec 0 défaillance (0.5 point)}
\lstinputlisting{Res/System0FCtrlVV.res}
\lstinputlisting{Res/System0FCtrlVV0F1I.res}
\lstinputlisting{Res/System0FCtrlVV0F2I.res}
%\lstinputlisting{Res/System0FCtrlVV0F3I.res}
%\lstinputlisting{Res/System0FCtrlVV0F4I.res}
\paragraph{Interprétation des résultats}
\begin{flushleft}
    On constate que pour un système utilisant le contrôleur CtrlVV avec 0 erreur, on constate que entre le contrôleur de base et la dernière itération on diminue de 153 le nombre d'état total
    et de  1355 le nombre de transition. De plus, on supprime entièrement les états puits et les états de niveaux critiques ce qui fait que nous n'avons plus de situation redouté. 
    De plus dans le  contrôleur de base, environ 29 \% des transitions sont des coups gagnants.
    A la dernière itération 71 \% des transitions sont gagnantes.
    
\end{flushleft}

\subsubsection{Avec 1 défaillance (0.5 point)}
\lstinputlisting{Res/System1FCtrlVV.res}
\lstinputlisting{Res/System1FCtrlVV1F1I.res}
\lstinputlisting{Res/System1FCtrlVV1F2I.res}
\lstinputlisting{Res/System1FCtrlVV1F3I.res}
\lstinputlisting{Res/System1FCtrlVV1F4I.res}
\paragraph{Interprétation des résultats}
\begin{flushleft}
    On constate que pour un système utilisant le contrôleur CtrlVV avec 1 erreur, on constate que entre le contrôleur de base et la dernière itération on diminue de 977 le nombre d'état total
    et de 7625 le nombre de transition. De plus, on supprime entièrement les états puits et les états de niveaux critiques ce qui fait que nous n'avons plus de situation redouté. 
    De plus dans le contrôleur de base, environ 22 \% des transitions sont des coups gagnants.
    A la dernière itération 53 \% des transitions sont gagnantes.
    
\end{flushleft}

\subsubsection{Avec 2 défaillances (0.5 point)}
\lstinputlisting{Res/System2FCtrlVV.res}
\lstinputlisting{Res/System2FCtrlVV2F1I.res}
\lstinputlisting{Res/System2FCtrlVV2F2I.res}
\lstinputlisting{Res/System2FCtrlVV2F3I.res}
%\lstinputlisting{Res/System2FCtrlVV2F4I.res}
\paragraph{Interprétation des résultats}
\begin{flushleft}
    On constate que pour un système utilisant le contrôleur CtrlVV avec 2 erreur, on constate que entre le contrôleur de base et la dernière itération on diminue de 2396 le nombre d'état total
    et de 15890 le nombre de transition. De plus, on supprime entièrement les états puits et les états de niveaux critiques ce qui fait que nous n'avons plus de situation redouté. 
    De plus dans le contrôleur de base, environ 14 \% des transitions sont des coups gagnants.
    A la dernière itération 25 \% des transitions sont gagnantes.
    
\end{flushleft}

\subsubsection{Avec 3 défaillances (0.5 point)}
\lstinputlisting{Res/System3FCtrlVV.res}
\lstinputlisting{Res/System3FCtrlVV3F1I.res}
\lstinputlisting{Res/System3FCtrlVV3F2I.res}
\lstinputlisting{Res/System3FCtrlVV3F3I.res}
%\lstinputlisting{Res/System3FCtrlVV3F4I.res}
\paragraph{Interprétation des résultats}
\begin{flushleft}
    On constate que pour un système utilisant le contrôleur CtrlVV avec 3 erreur, on constate que entre le contrôleur de base et la dernière itération on diminue de 2887 le nombre d'état total
    et de 18772 le nombre de transition. De plus, on supprime entièrement les états puits et les états de niveaux critiques ce qui fait que nous n'avons plus de situation redouté. 
    De plus dans le contrôleur de base, environ 13 \% des transitions sont des coups gagnants.
    A la dernière itération 25 \% des transitions sont gagnantes.
    
\end{flushleft}

\subsection{Calcul des contrôleurs optimisés (2 points)}
% A COMPLETER en analysant les contrôleurs optimisés obtenus.
\paragraph{Avec 0 défaillance}\ \\
\lstinputlisting{Res/System0FCtrlVV0F2I_Opt.res}

\paragraph{Avec 1 défaillance}\ \\
\lstinputlisting{Res/System1FCtrlVV1F4I_Opt.res}

\paragraph{Avec 2 défaillances}\ \\
\lstinputlisting{Res/System2FCtrlVV2F3I_Opt.res}

\paragraph{Avec 3 défaillances}\ \\
\lstinputlisting{Res/System3FCtrlVV3F3I_Opt.res}

\section{Conclusion (2 points)}
% A COMPLETER

\end{document}
